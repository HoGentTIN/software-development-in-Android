\chapter{User Interfaces}
Android gives some key components that can be used to create user interface. All the Android user interface are built using these key components:

\begin{description}
	\item[View] It is the base class for all visual components (control and widgets). All the controls present in an android app are derived
	from this class. A View is an object that draws something on a smartphone screen and enables an user to interact with it.
	\item[Viewgroup] A ViewGroup can contain one or more Views and defines how these Views are placed in the user interface
	(these are used along with Android Layout managers.
	\item[Fragments] s Introduced from API level 11, this component encapsulates a single piece of UI interface. They are very useful
	when we have to create and optimize our app user interface for multiple devices or multiple screen size.
	\item[Activities] Usually an Android app consists of several activities that exchange data and information. An Activity takes
	care of creating the user interface.
\end{description}

If we analyze in more detail an Android user interface, we can notice that it has an hierarchical structure where at the root there’s
a ViewGroup. A ViewGroup behaves like an invisible container where single views are placed following some rules. We
can combine a ViewGroup with another ViewGroup to have more control on how views are located. We have to remember that complex user interfaces require more time to render it. \textbf{Therefore, for better performance we should create
simple UIs.} Additionally, a clean interface helps user to have a better experience when using our app.

Before we continue we need to define some key concepts:

\begin{description}
	\item[Screen size] It is the physical screen or in other words, the real dimension of our device screen.
	\item[Density] It is the number of pixels in a given area. Usually we consider dot per inch (dpi). This is a measure of the screen
	quality.
	\item[Orientation] This is how the screen is oriented. It can be landscape or portrait.
	\item[Density independent pixel] This is a new pixel unit measure introduced by Android. It is called dp. One dp is equivalent at
	one pixel at a 160dpi screen. We should use dp unit in our measures when creating an UI, at the runtime the system takes care of
	converting it into a real pixel.
\end{description}

From the screen size point of view, Android groups the devices in four areas,small, normal, large and extra large (xlarge),
depending on the actual screen dimension expressed in inches. From the dpi point of view, on the other hand, we can group
devices in: ldpi (low dpi), mdpi (medium dpi), hdpi (high dpi), xhdpi (extra high dpi) and lately xxhdpi. This is important when
we use drawables (i.e bitmaps), because we have to create several images according to the different screen resolution.

There are some best practices regarding the user interface:

\begin{framed}
	
	
	\begin{enumerate}
		\item Don’t use fixed dimensions expressed in pixel, instead we should use dp.
		\item Provide several layout structures for different screen size, we can do it creating several layout files.
		\item Provide several bitmap with different resolution for different screen resolutions. 
	\end{enumerate}

\end{framed}

\section{Material Design Basics}