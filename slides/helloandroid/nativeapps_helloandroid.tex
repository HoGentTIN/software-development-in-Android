%%----------------------------------------------------------------------------
%% Presentatie HoGent Bedrijf en Organisatie
%%----------------------------------------------------------------------------
%% Auteur: Bert Van Vreckem [bert.vanvreckem@hogent.be]

\documentclass{beamer}

%==============================================================================
% Aanloop
%==============================================================================

%---------- Packages ----------------------------------------------------------

\usepackage{graphicx,multicol}
\usepackage{comment,enumerate,hyperref}
\usepackage{amsmath,amsfonts,amssymb}
\usepackage{tikz}
\usepackage[dutch]{babel}
\usepackage[utf8]{inputenc}
\usepackage{multirow}
\usepackage{eurosym}
\usepackage{listings}
\usepackage[T1]{fontenc}
\usepackage{lmodern}
\usepackage{textcomp}
\usepackage{framed}
\usepackage{wrapfig}

%---------- Configuratie ------------------------------------------------------

\usetikzlibrary{arrows,shapes,backgrounds,positioning,shadows}

\usetheme{hogent}

%---------- Commando-definities -----------------------------------------------

\newcommand{\tabitem}{~~\llap{\textbullet}~~}

%---------- Info over de presentatie ------------------------------------------

\title[Intro]{Native Apps 1: Hello Android}
\author{Karine Samyn, Jens Buysse}
\date{AJ 2017-2018}

%==============================================================================
% Inhoud presentatie
%==============================================================================

\begin{document}

%---------- Front matter ------------------------------------------------------

% Dia met het HoGent logo
\HoGentLogo

% Titeldia met faculteitslogo
\titleframe

%---------- Inhoud ------------------------------------------------------------

\begin{frame}
  \frametitle{Inhoud}

  \tableofcontents
\end{frame}

%---------- Back matter -------------------------------------------------------

\section{Overzicht cursus}

\sectionframe{Dit is een ondershrift}

\begin{frame}{Welkom op de cursus}
	Dit is inhoud
\end{frame}

\begin{frame}[fragile]{Welkom op de cursus}

	\begin{lstlisting}[language=java]
		System.out.println();
		int testk  = 10;
	\end{lstlisting}

\end{frame}





\end{document}
