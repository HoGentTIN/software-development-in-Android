\chapterimage{images/intents/intentchapterimage.jpg}

\chapter{Intents and Broadcastreceivers}


\begin{example}
	In this example we see an application which is able to convert speech into text.
	You can find the application here \cite{Buysse18}.
	This activity contains a button, which will create an special intent launching an Activity which is able to listen to speech, convert it to text and return its results.
	You will need an actual Android Device to test this. 
	
	The application is also  able to provide the user with a list of extra actions which he is able to perform. The choices are the following:
	\begin{itemize}
		\item Open a web-site with a given url (validate the URL)
		\item Open the contacts
		\item Open another Activity
		\item Open the dialer
		\item Search google
	\end{itemize}
	
\end{example}


\section{Intent}
Intents are objects which you can use for the following actions:


\begin{itemize}
	\item Explicitly start a particular Service or Activity using its class name (already seen in previous lessons)
	\item Implicitly start a particular Service or Activity
	\item Start an Activity or Service to perform an action with (or on) a particular piece of data
	\item Broadcast that an event has occurred
\end{itemize}

The two most important pieces of an Intent are the action and what Android refers to as the data.
If you were to create an Intent combining ACTION\_VIEW with a content Uri of https://google.com, and pass that Intent to Android via startActivity(), Android would know to find and open an activity capable of viewing that resource.


There are other criteria you can place inside an Intent
\begin{description}
	\item[Categories] A string containing additional information about the kind of component that should handle the intent.
	Any number of category descriptions can be placed in an intent, but most intents do not require a category.
	Your “main” activity will be in the LAUNCHER category, indicating it should show up on the launcher menu.
	\item[A MIME type]  indicating the type of resource you want to operate on.
	\item[Extras] which is a Bundle of other information you want to pass along to the receiver with the Intent, that the recipient might want to take advantage
\end{description}

If you specify the target component in your Intent , Android has no doubt where the Intent is supposed to be routed to and it will launch the named activity or application component.
This might be OK if the target recipient (e.g., the activity to be started) is in your application.

\begin{framed}
		This way of starting components (explicit) is definitely not recommended for invoking functionality in other applications.
		Component name are considered private to the application and are subject to change.
\end{framed}

There are two types of intents: explicit and implicit.

\subsection{Explicit Intent}
\textbf{An explicit intent} is one that you use to launch a specific app component, such as a particular activity or service in your app.
To create an explicit intent, define the component name for the Intent object, all other Intent properties are optional.

For example, if you want to start another activity from your application you could use the following code. 

\lstinputlisting[firstline=89,lastline=92,language=Kotlin, caption={Starting an explicit intent}, label=code:explicitIntent]{srccode/intents/fragments/MainActivityFragment.kt}

The Intent is built inside the MainActivity and not outside. You cannot make it wrong and accidentally forget an extra. 

\lstinputlisting[firstline=23,lastline=32,language=Kotlin, caption={Generating the intent in the Activity.}, label=code:explicitIntent]{srccode/intents/activities/MainActivity.kt}

\subsection{Implicit intent}
An implicit intent specifies an action that can invoke any app on the device able to perform the action.
Using an implicit intent is useful when your app cannot perform the action, but other apps probably can and you'd like the user to pick which app to use.

For example, if you need a contact from the Contact on your user's phone you could generate an Intent and start an activity to get it. 

\lstinputlisting[firstline=58,lastline=61,language=Kotlin, caption={Starting an implicit intent to find a contact from the contact app }, label=code:explicitIntent]{srccode/intents/fragments/MainActivityFragment.kt}


\begin{framed}
It's good practice to determine if your call will resolve to an Activity before calling startActivity. In our example application we have created the method checkForCompatibility.
\end{framed}

\lstinputlisting[firstline=121,lastline=144,language=Kotlin, caption={Check if there is an Activity which is able to handle the intent}, label=code:explicitIntent]{srccode/intents/fragments/MainActivityFragment.kt}

\subsection{Allow to start your Activity from another app}

\begin{figure}
	\includegraphics[width=\textwidth]{images/intents/intentresolution.png}
	\caption{Android OS uses filters to pinpoint the set of Activities, Services, and Broadcast receivers that can handle the Intent with help of specified set of action, categories, data scheme associated with an Intent.
		You will use <intent-filter> element in the manifest file to list down actions, categories and data types associated with any activity, service, or broadcast receiver.}
	\label{fig:intentresolution}
\end{figure}

To allow other apps to start your activity, you need to add an intent-filter element in your manifest file for the corresponding activity element.
In order to properly define which intents your activity can handle, each intent filter you add should be as specific as possible in terms of the type of action and data the activity accepts.
The system may send a given Intent to an activity if that activity has an intent filter that fulfils the following criteria of the Intent object:

\begin{itemize}
	\item Action : A string naming the action to perform.
	\item Data : A description of the data associated with the intent.
	\item Category: Provides an additional way to characterize the activity handling the intent, usually related to the user gesture or location from which it's started.
\end{itemize}

For example, here's an activity declaration with an intent filter to receive an ACTION\_SEND intent when the data type is text:

\lstinputlisting[firstline=48,lastline=54,language=kxml, caption={An example intent filter}, label=code:explicitIntent]{srccode/intents/explicit.kt}

\section{Broadcastreceiver}
So far, you’ve looked at using Intents to start new application components, but you can also use Intents to broadcast messages anonymously between components via the sendBroadcast() method.
As a system-level message-passing mechanism, Intents are capable of sending structured messages across process boundaries.
As a result, you can implement BroadcastReceivers to listen for, and respond to, these Broadcast Intents within your applications.

Within your application, construct the Intent you want to broadcast and call sendBroadcast() to send it.
Set the action, data, and category of your Intent in a way that lets BroadcastReceivers accurately determine their interest.

\subsection{Receiving a broadcast (without the manifest)}
To receive such a broadcast in an activity (or a fragment), you will need to do four things.

\begin{enumerate}
	\item You will need to create an instance of your own subclass of BroadcastReceiver.
	The only method you need to (or should) implement is onReceive(), which will be passed the Intent that was broadcast, along with a Context object that, in this case, you will typically ignore. 
	\item Second, you will need to create an instance of an IntentFilter object, describing	the sorts of broadcasts you want to receive. 
	Most of these filters are set up to watch for a single broadcast Intent action.
	\item You will need to call registerReceiver(), typically from onStart() or onResume() of your activity or fragment, supplying your BroadcastReceiver and your IntentFilter.
	\item You will need to call unregisterReceiver(), typically from onStop() or onPause() of your activity or fragment, supplying the same BroadcastReceiver instance you provided	to registerReceiver().
\end{enumerate}

\subsection{Receiving a broadcast (with the manifest)}
You can also tell Android about broadcasts you wish to receive by adding a <receiver> element to your manifest, identifying the class that implements your BroadcastReceiver (via the android:name attribute), plus an <intent-filter> that describes the broadcast(s) you wish to receive:

\lstinputlisting[firstline=57,lastline=61,language=kxml, caption={Receiving a Broadcast with the manifest}, label=code:explicitIntent]{srccode/intents/explicit.kt}

The good news is that this BroadcastReceiver will be available for broadcasts occurring at any time.
There is no assumption that you have an activity already running that called registerReceiver().
The bad news is that the instance of the BroadcastReceiver used by Android to process a broadcast will live for only so long as it takes to execute the onReceive() method.
At that point, the BroadcastReceiver is discarded.
Hence, it is not safe for a manifest-registered BroadcastReceiver to do anything that needs to run after onReceive() itself completes, such as forking a thread.
After all, Android may well terminate the process within milliseconds, if there is no other running component in the process.
Moreover, onReceive() is called on the main application thread and may freeze your UI.