% Cursus Onderzoekstechnieken
%
% Genereer PDF-versie met volgende procedure:
% 
% 1) latexmk -pdf "cursus-onderzoekstechnieken"
% 2) biber "cursus-onderzoekstechnieken"
% 3) latexmk -pdf "cursus-onderzoekstechnieken"
%
\documentclass[11pt,fleqn,a4paper]{book}

\input{structure}


\author{dr. Jens Buysse, Harm De Weirdt}
\title{Native Apps 1: Android}
\date{2017-2018}

% Generate the glossary
\makeglossaries



\begin{document}

\thetitlepage

%----------------------------------------------------------------------------------------
%	COPYRIGHT PAGE
%----------------------------------------------------------------------------------------

\newpage
~\vfill
\thispagestyle{empty}

\noindent Copyright \copyright\ 2017-2017 Jens Buysse\\ % Copyright notice

\noindent \textsc{www.hogent.be}\\ % URL

\noindent \textit{Generated on \today} % Printing/edition date

%----------------------------------------------------------------------------------------
%	TABLE OF CONTENTS
%----------------------------------------------------------------------------------------

\chapterimage{images/chapterhead1.jpg}

\tableofcontents % Print the table of contents itself

%Print the glossary
\printglossaries

\cleardoublepage % Forces the first chapter to start on an odd page so it's on the right

\setlength{\parindent}{0pt}

\includecomment{solution}
%\excludecomment{solution}
At this moment, there is not realy anybody I want to thank as this is the first time I'm publishing this course. Maybe I could thank my girlfriend, but that would be kind of strange as she has done nothing to help me write this document. 

Maybe later, you the student reading and studying this course, could be thanked if you helped by giving feedback, correcting and showing some cool new things in Android. 

But for the time being, I wish you a lot of fun reading and studying this course!

\begin{flushright}
	-- Jens Buysse
\end{flushright}

\chapter{About this course}

\section{Introduction}
This course is an introduction into the wonderful world of Android. The aim of this course is not to generate specialized Android developers, but to introduce you to the general concepts of the Android framework and provide you to tools to find your way in this fragmented world. 

\section{ECTS}
ECTS is an acronym for European Credit Transfer and Accumulation System. This stipulates every final attainment level you should have when finishing this course. These can be found \href{here}{https://bamaflexweb.hogent.be/BMFUIDetailxOLOD.aspx?a=87232\&b=5\&c=1}. 

\section{Course Material}
The course is taught using the following materials:

\begin{itemize}
	\item These course notes
	\item Book \textit{Professional Android 4}  \cite{Reto2017}
	\item Book  \textit{The Busy Coder's Guide to Android Development} \cite{murphymarkl.2017}
	\item External resources and references (these course notes, Chamilo links \dots)
\end{itemize}

\section{Course organisation}
The lessons are organized in 12 lessons, where we will go over the theoretical aspects the first part of the lessen, where the second part will be used to put the things learned into practice during an exercise. Normally you will not be able to complete the exercise in one hour so it is up to you to complete the exercises at home. If you encounter any difficulties  you are welcome to ask questions in the next lesson of via the forum on Chamilo. We don't encourage sending personal email: you are
not alone in having difficulty with certain topics. Email is not the best way for us to help you with it.

When asking question, take care to format your questions: make it reproducible. Follow these guidelines: \href{https://stackoverflow.com/help/how-to-ask}{https://stackoverflow.com/help/how-to-ask}

\section{Prior knowledge}
This course assumes that you know Java at this point. If you do not, you will need to learn Java before you go much further. You do not need to know everything about Java, as Java is vast. Rather, focus on:

\begin{itemize}
	\item Language fundamentals (flow control, etc.)
	\item Classes and objects
	\item Methods and data members
	\item Public, private, and protected
	\item Static and instance scope
	\item Exceptions
	\item Threads
	\item Collections
	\item Generics
	\item File I/O
	\item Reflection
	\item Interfaces
\end{itemize}

\section{Examination}
Both for the first as for the second exam period, you will be examined via an oral exam. You will be handed a set of questions which you will have some time to prepare for. Some of the questions will be regarding the exercises you have made during the year, so you are obliged to have made all the exercises. \textbf{If your question is one regarding an exercise you have not made, you will receive a 0 on that question.}

Working together on the exercises is encouraged, but make sure that
You have programmed and written the code yourself. You will get some hard questions and it is very difficult to answer when you have not written the code yourself.
Document your code well! Make sure you remember why you have written something the way you have.



\newacronym{utc}{UTC}{Coordinated Universal Time}

\newacronym{apk}{APK}{Android Application Package }
\newacronym{jdk}{JDK}{Java Development Kit}
\newacronym{repl}{REPL}{Read–Eval–Print Loop}



\chapterimage{images/hellochapterhead.jpg}

\chapter{Hello Android}
Android\cite{Todd2017} is a mobile operating system which is found on a variety of modern devices, the most popular being smartphones. On top of that, you will also find Android on tablets, TV streaming boxes and other portable gadgets.

Android is basically a piece of software which allows your hardware to function. The Android OS gives you access to apps, including many of Google's own creation. These allow you to look for information on the web, play music and videos, check your location on a map, take photos using your device's camera and plenty more besides.

\subsection{Open source Android}
Android has open source roots. The project began under Android, Inc. in 2005, which Google bought two years later. That same year, Google and several other companies formed  \footnote{The Open Handset Alliance is a group of 84 technology and mobile companies who have come together to accelerate innovation in mobile and offer consumers a richer, less expensive, and better mobile experience. They have developed Android, the first complete, open, and free mobile platform.}{Open Handset Alliance} \cite{alliance}, with Android being the primary piece of software this consortium is built on.

Android is based on the Linux kernel, and like that complex piece of code, most parts are open source with a few binary blobs included to make things work with certain hardware. The core Android platform, known as the Android Open Source Project (AOSP), is available for anyone to do with what they wish.

But is it realy open source ? For the most part, Google develops Android. Once or twice a year, the company dumps  new code over a metaphorical wall that tinkerers and hardware makers use to put in their own code. This is in contrast with many other well-known open source projects : they typically seek more involvement from the broader community. Red Hat may fund a good portion of the work that goes into GNOME, but developers from all over the world contribute code. By comparison, Android comes off as entirely a Google product.



\section{Android version \& updates}
Google is constantly working on new versions of the Android software. These releases are infrequent; at the moment Google is releasing a big Android update once a year.

Versions usually come with a numerical code and a name that’s so far been themed after sweets and desserts, running in alphabetical order.

\begin{description}
	\item[Android 1.5]  Cupcake
	\item[Android 1.6]  Donut
	\item[Android 2.1]  Eclair
	\item[Android 2.2]  Froyo
	\item[Android 2.3]  Gingerbread
	\item[Android 3.2 Honeycomb]  - The first OS design specifically for tablets, launching on the Motorola Xoom
	\item[Android 4.0 Ice Cream Sandwich] : The first OS to run on smartphones and tablets, ending the 2.X naming convention.
	\item[Android 4.1 Jelly Bean]  Launched on the Google Nexus 7 tablet by Asus
	\item[Android 4.2]  Jelly Bean: Arrived on the LG Nexus 4
	\item[Android 4.3]  Jelly Bean
	\item[Android 4.4 KitKat]  Launched on the LG Nexus 5
	\item[Android 5.0 Lollipop]  Launched on the Motorola Nexus 6 and HTC Nexus 9
	\item[Android 6.0 Marshmallow]  Launched on the LG Nexus 5X and Huawei Nexus 6P
	\item[Android 7.0 Nougat] 
	\item[Android 7.1 Nougat]  Launched on the HTC-made Google Pixel and Pixel XL
	A\item[Android 8.0 Oreo] Rumoured to be launching on the Google Pixel 2 and Pixel XL 2
\end{description}

\subsection{API Levels}
The core Android development team tries very hard to ensure forwards and backwards compatibility. An app you write today should work unchanged on future versions of Android (forwards compatibility), albeit perhaps missing some features or working in some sort of “compatibility mode”. And there are well-trod paths for how to create apps that will work both on the latest and on previous versions of Android (backwards compatibility).

To help us keep track of all the different OS versions that matter to us as developers, Android has API levels. A new API level is defined when an Android version ships that contains changes that affect developers. When you create an emulator to test your app, you will indicate what API level that emulator should emulate. When you distribute your app, you will indicate the oldest API level your app supports, so the app is not installed on older devices.

\section{The Android Software stack}
The Android software stack is a Linux kernel and a collection of C/C++ libraries exposed through an application framework that provides services for, and management of, the run time and applications. It consists of several components \cite{google2017}:

\begin{itemize}
	\item \textbf{Linux kernel} The Android Runtime (ART) relies on the Linux kernel for underlying functionalities such as threading and low-level memory management.
	Using a Linux kernel allows Android to take advantage of key security features and allows device manufacturers to develop hardware drivers for a well-known kernel.
	\item \textbf{Hardware Abstraction Layer (HAL)} The hardware abstraction layer (HAL) provides standard interfaces that expose device hardware capabilities to the higher-level Java API framework. The HAL consists of multiple library modules, each of which implements an interface for a specific type of hardware component, such as the camera or bluetooth module. When a framework API makes a call to access device hardware, the Android system loads the library module for that hardware component.
	\item \textbf{Android Runtime} Each app runs in its own process and with its own instance of the Android Runtime (ART). ART is written to run multiple virtual machines on low-memory devices by executing \textbf{DEX files}, a bytecode format designed specially for Android that's optimized for minimal memory footprint. It uses
	\begin{itemize}
		\item Ahead-of-time (AOT) and just-in-time (JIT) compilation \cite{}
		\item Optimized garbage collection (GC)
	\end{itemize}
	Prior to Android version 5.0 (API level 21), Dalvik was the Android runtime. If your app runs well on ART, then it should work on Dalvik as well, but the reverse may not be true.
	\item \textbf{Native C/C++ Libraries} Many core Android system components and services, such as ART and HAL, are built from native code that require native libraries written in C and C++. 
	\item \textbf{Java API Framework} The entire feature-set of the Android OS is available to you through APIs written in the Java language. These APIs form the building blocks you need to create Android apps by simplifying the reuse of core, modular system components and services.
	\item \textbf{System Apps} Android comes with a set of core apps for email, SMS messaging, calendars, internet browsing, contacts, and more. Apps included with the platform have no special status among the apps the user chooses to install. So a third-party app can become the user's default web browser, SMS messenger, or even the default keyboard (some exceptions apply, such as the system's Settings app).
	
	
	
	
\end{itemize}


\begin{figure}[ht]
	\centering
	\includegraphics[width=\textwidth]{images/hello/android-stack.png}
	\label{fig:stack}
	\caption{Android is an open source, Linux-based software stack created for a wide array of devices and form factors. The following diagram shows the major components of the Android platform. Figure from \cite{google2017}}
\end{figure}



\chapterimage{images/kotlin/kotlinprogramminglanguage.png}

\chapter{Hello Kotlin}
\chapterimage{images/activities.jpg} % Chapter heading image

\chapter{Activities and Their Lifecycles}
\chapter{User Interfaces}
Android gives some key components that can be used to create user interface. All the Android user interface are built using these key components:

\begin{description}
	\item[View] It is the base class for all visual components (control and widgets). All the controls present in an android app are derived
	from this class. A View is an object that draws something on a smartphone screen and enables an user to interact with it.
	\item[Viewgroup] A ViewGroup can contain one or more Views and defines how these Views are placed in the user interface
	(these are used along with Android Layout managers.
	\item[Fragments] s Introduced from API level 11, this component encapsulates a single piece of UI interface. They are very useful
	when we have to create and optimize our app user interface for multiple devices or multiple screen size.
	\item[Activities] Usually an Android app consists of several activities that exchange data and information. An Activity takes
	care of creating the user interface.
\end{description}

If we analyze in more detail an Android user interface, we can notice that it has an hierarchical structure where at the root there’s
a ViewGroup. A ViewGroup behaves like an invisible container where single views are placed following some rules. We
can combine a ViewGroup with another ViewGroup to have more control on how views are located. We have to remember that complex user interfaces require more time to render it. \textbf{Therefore, for better performance we should create
simple UIs.} Additionally, a clean interface helps user to have a better experience when using our app.

Before we continue we need to define some key concepts:

\begin{description}
	\item[Screen size] It is the physical screen or in other words, the real dimension of our device screen.
	\item[Density] It is the number of pixels in a given area. Usually we consider dot per inch (dpi). This is a measure of the screen
	quality.
	\item[Orientation] This is how the screen is oriented. It can be landscape or portrait.
	\item[Density independent pixel] This is a new pixel unit measure introduced by Android. It is called dp. One dp is equivalent at
	one pixel at a 160dpi screen. We should use dp unit in our measures when creating an UI, at the runtime the system takes care of
	converting it into a real pixel.
\end{description}

From the screen size point of view, Android groups the devices in four areas,small, normal, large and extra large (xlarge),
depending on the actual screen dimension expressed in inches. From the dpi point of view, on the other hand, we can group
devices in: ldpi (low dpi), mdpi (medium dpi), hdpi (high dpi), xhdpi (extra high dpi) and lately xxhdpi. This is important when
we use drawables (i.e bitmaps), because we have to create several images according to the different screen resolution.

There are some best practices regarding the user interface:

\begin{framed}
	
	
	\begin{enumerate}
		\item Don’t use fixed dimensions expressed in pixel, instead we should use dp.
		\item Provide several layout structures for different screen size, we can do it creating several layout files.
		\item Provide several bitmap with different resolution for different screen resolutions. 
	\end{enumerate}

\end{framed}

\section{Material Design Basics}
\chapterimage{images/fragments/fragments.jpg} % Chapter heading image

\chapter{Fragments}
Fragments are an optional layer you can put between your activities and your
widgets, designed to help you reconfigure your activities to support screens both
large (e.g., tablets) and small (e.g., phones). 

If you regard Android as an MVC architecture, fragments and
activities combine to be the controller layer. Fragments serve as a local controller, focused on their set of widgets, populating them from model data, and handling
their events. Activities will serve as more of an orchestration layer, handling cross-
fragment communications (e.g., a click in Fragment A needs to cause a change in what is displayed in Fragment B).

\section{Design Philopsophy of fragments}
Android introduced fragments in Android 3.0 (API level 11), primarily to support more dynamic and flexible UI designs on large screens, such as tablets. Because a tablet's screen is much larger than that of a handset, there's more room to combine and interchange UI components. Fragments allow such designs without the need for you to manage complex changes to the view hierarchy. By dividing the layout of an activity into fragments, you become able to modify the activity's appearance at runtime and preserve those changes in a back stack that's managed by the activity.

You should design each fragment as a modular and reusable activity component. That is, because each fragment defines its own layout and its own behaviour with its own life cycle callbacks, you can include one fragment in multiple activities, so you should design for reuse and \textbf{avoid directly manipulating one fragment from another fragment}. This is especially important because a modular fragment allows you to change your fragment combinations for different screen sizes.

\section{How to work with activities and fragments}
To explain the use of fragments we will be looking at an example implementation from  \cite{murphymarkl.2017}. You can find the link \href{https://github.com/commonsguy/cw-omnibus/tree/master/Fragments/Static}{here}.

\begin{framed}
	When implementing the exercises for this course you have to use github. Off course you know you do not put every file of your project on the repository. Look how the author of \cite{murphymarkl.2017} have done this on their repository. 
\end{framed}
\begin{enumerate}
	\item To create a fragment, you must create a subclass of Fragment (or an existing subclass of it). The Fragment class has code that looks a lot like an Activity. It contains callback methods similar to an activity, such as onCreate(), onStart(), onPause(), and onStop().
	\item To provide a layout for a fragment, you must implement the onCreateView() callback method, which the Android system calls when it's time for the fragment to draw its layout. Your implementation of this method must return a View that is the root of your fragment's layout.
	\item To add the fragment:
	\begin{enumerate}
		\item  Declare the fragment inside the activity's layout file.
		In this case, you can specify layout properties for the fragment as if it were a view. 
		\item Or, programmatically add the fragment to an existing ViewGroup. To make fragment transactions in your activity (such as add, remove, or replace a fragment), you must use APIs from FragmentTransaction. A simple example can be found \href{https://github.com/commonsguy/cw-omnibus/tree/master/Fragments/Dynamic}{here}.
	\end{enumerate}
\end{enumerate}

\section{Fragment Life cycle}
Fragments have lifecycle methods, just like activities do. In fact, they support most
of the same lifecycle methods as activities:
\begin{itemize}
	\item onCreate()
	\item onStart() 
	\item onResume()
	\item onPause()
	\item onStop()
	\item onDestroy()
\end{itemize}
Almost the same rules apply for fragments as do for activities. It is up to the reader to look when each life cycle method is called and when to use it. 

\section{Communication with other Fragments and Activities}




\chapterimage{images/intents/intentchapterimage.jpg}

\chapter{Intents and Broadcastreceivers}


\begin{example}
	In this example we see an application which is able to convert speech into text.
	You can find the application here \cite{Buysse18}.
	This activity contains a button, which will create an special intent launching an Activity which is able to listen to speech, convert it to text and return its results.
	You will need an actual Android Device to test this. 
	
	The application is also  able to provide the user with a list of extra actions which he is able to perform. The choices are the following:
	\begin{itemize}
		\item Open a web-site with a given url (validate the URL)
		\item Open the contacts
		\item Open another Activity
		\item Open the dialer
		\item Search google
	\end{itemize}
	
\end{example}


\section{Intent}
Intents are objects which you can use for the following actions:


\begin{itemize}
	\item Explicitly start a particular Service or Activity using its class name (already seen in previous lessons)
	\item Implicitly start a particular Service or Activity
	\item Start an Activity or Service to perform an action with (or on) a particular piece of data
	\item Broadcast that an event has occurred
\end{itemize}

The two most important pieces of an Intent are the action and what Android refers to as the data.
If you were to create an Intent combining ACTION\_VIEW with a content Uri of https://google.com, and pass that Intent to Android via startActivity(), Android would know to find and open an activity capable of viewing that resource.


There are other criteria you can place inside an Intent
\begin{description}
	\item[Categories] A string containing additional information about the kind of component that should handle the intent.
	Any number of category descriptions can be placed in an intent, but most intents do not require a category.
	Your “main” activity will be in the LAUNCHER category, indicating it should show up on the launcher menu.
	\item[A MIME type]  indicating the type of resource you want to operate on.
	\item[Extras] which is a Bundle of other information you want to pass along to the receiver with the Intent, that the recipient might want to take advantage
\end{description}

If you specify the target component in your Intent , Android has no doubt where the Intent is supposed to be routed to and it will launch the named activity or application component.
This might be OK if the target recipient (e.g., the activity to be started) is in your application.

\begin{framed}
		This way of starting components (explicit) is definitely not recommended for invoking functionality in other applications.
		Component name are considered private to the application and are subject to change.
\end{framed}

There are two types of intents: explicit and implicit.

\subsection{Explicit Intent}
\textbf{An explicit intent} is one that you use to launch a specific app component, such as a particular activity or service in your app.
To create an explicit intent, define the component name for the Intent object, all other Intent properties are optional.

For example, if you want to start another activity from your application you could use the following code. 

\lstinputlisting[firstline=89,lastline=92,language=Kotlin, caption={Starting an explicit intent}, label=code:explicitIntent]{srccode/intents/fragments/MainActivityFragment.kt}

The Intent is built inside the MainActivity and not outside. You cannot make it wrong and accidentally forget an extra. 

\lstinputlisting[firstline=23,lastline=32,language=Kotlin, caption={Generating the intent in the Activity.}, label=code:explicitIntent]{srccode/intents/activities/MainActivity.kt}

\subsection{Implicit intent}
An implicit intent specifies an action that can invoke any app on the device able to perform the action.
Using an implicit intent is useful when your app cannot perform the action, but other apps probably can and you'd like the user to pick which app to use.

For example, if you need a contact from the Contact on your user's phone you could generate an Intent and start an activity to get it. 

\lstinputlisting[firstline=58,lastline=61,language=Kotlin, caption={Starting an implicit intent to find a contact from the contact app }, label=code:explicitIntent]{srccode/intents/fragments/MainActivityFragment.kt}


\begin{framed}
It's good practice to determine if your call will resolve to an Activity before calling startActivity. In our example application we have created the method checkForCompatibility.
\end{framed}

\lstinputlisting[firstline=121,lastline=144,language=Kotlin, caption={Check if there is an Activity which is able to handle the intent}, label=code:explicitIntent]{srccode/intents/fragments/MainActivityFragment.kt}

\subsection{Allow to start your Activity from another app}

\begin{figure}
	\includegraphics[width=\textwidth]{images/intents/intentresolution.png}
	\caption{Android OS uses filters to pinpoint the set of Activities, Services, and Broadcast receivers that can handle the Intent with help of specified set of action, categories, data scheme associated with an Intent.
		You will use <intent-filter> element in the manifest file to list down actions, categories and data types associated with any activity, service, or broadcast receiver.}
	\label{fig:intentresolution}
\end{figure}

To allow other apps to start your activity, you need to add an intent-filter element in your manifest file for the corresponding activity element.
In order to properly define which intents your activity can handle, each intent filter you add should be as specific as possible in terms of the type of action and data the activity accepts.
The system may send a given Intent to an activity if that activity has an intent filter that fulfils the following criteria of the Intent object:

\begin{itemize}
	\item Action : A string naming the action to perform.
	\item Data : A description of the data associated with the intent.
	\item Category: Provides an additional way to characterize the activity handling the intent, usually related to the user gesture or location from which it's started.
\end{itemize}

For example, here's an activity declaration with an intent filter to receive an ACTION\_SEND intent when the data type is text:

\lstinputlisting[firstline=48,lastline=54,language=kxml, caption={An example intent filter}, label=code:explicitIntent]{srccode/intents/explicit.kt}

\section{Broadcastreceiver}
So far, you’ve looked at using Intents to start new application components, but you can also use Intents to broadcast messages anonymously between components via the sendBroadcast() method.
As a system-level message-passing mechanism, Intents are capable of sending structured messages across process boundaries.
As a result, you can implement BroadcastReceivers to listen for, and respond to, these Broadcast Intents within your applications.

Within your application, construct the Intent you want to broadcast and call sendBroadcast() to send it.
Set the action, data, and category of your Intent in a way that lets BroadcastReceivers accurately determine their interest.

\subsection{Receiving a broadcast (without the manifest)}
To receive such a broadcast in an activity (or a fragment), you will need to do four things.

\begin{enumerate}
	\item You will need to create an instance of your own subclass of BroadcastReceiver.
	The only method you need to (or should) implement is onReceive(), which will be passed the Intent that was broadcast, along with a Context object that, in this case, you will typically ignore. 
	\item Second, you will need to create an instance of an IntentFilter object, describing	the sorts of broadcasts you want to receive. 
	Most of these filters are set up to watch for a single broadcast Intent action.
	\item You will need to call registerReceiver(), typically from onStart() or onResume() of your activity or fragment, supplying your BroadcastReceiver and your IntentFilter.
	\item You will need to call unregisterReceiver(), typically from onStop() or onPause() of your activity or fragment, supplying the same BroadcastReceiver instance you provided	to registerReceiver().
\end{enumerate}

\subsection{Receiving a broadcast (with the manifest)}
You can also tell Android about broadcasts you wish to receive by adding a <receiver> element to your manifest, identifying the class that implements your BroadcastReceiver (via the android:name attribute), plus an <intent-filter> that describes the broadcast(s) you wish to receive:

\lstinputlisting[firstline=57,lastline=61,language=kxml, caption={Receiving a Broadcast with the manifest}, label=code:explicitIntent]{srccode/intents/explicit.kt}

The good news is that this BroadcastReceiver will be available for broadcasts occurring at any time.
There is no assumption that you have an activity already running that called registerReceiver().
The bad news is that the instance of the BroadcastReceiver used by Android to process a broadcast will live for only so long as it takes to execute the onReceive() method.
At that point, the BroadcastReceiver is discarded.
Hence, it is not safe for a manifest-registered BroadcastReceiver to do anything that needs to run after onReceive() itself completes, such as forking a thread.
After all, Android may well terminate the process within milliseconds, if there is no other running component in the process.
Moreover, onReceive() is called on the main application thread and may freeze your UI.
\chapterimage{images/recycler/recycling.jpg}
\chapter{Recyclerview}
In 2014, Google released RecyclerView , via the Android Support package.
Developers can add the recyclerview-v7 to their projects and use RecyclerView. In this chapter, we will review RecyclerView from the ground up, starting with basic
operation.

Before you are able to use the recylceview,  open build.gradle (app) and add the dependencies needed.

\begin{xml}
    compile 'com.android.support:recyclerview-v7:+'
	compile 'com.android.support:cardview-v7:+'
\end{xml}

\section{What is a recyclerview}
The RecyclerView is a new ViewGroup that is prepared to render any adapter-based view. It is supposed to be the successor of ListView and GridView, and it can be found in the latest support-v7 version. One of the reasons is that RecyclerView has a more extensible framework, especially since it provides the ability to implement both horizontal and vertical layouts. Use the RecyclerView widget when you have data collections whose elements change at runtime based on user action or network events.

To use a recyclerview, you will need the following things:
\begin{enumerate}
	\item RecyclerView.Adapter - To handle the data collection and bind it to the view
	\item A layout file defining the row layout in the recyclerview. Just a note here: when you create the item layout of the RecyclerView don’t forget to add the following lines in the ViewGroup container of the layout. This lines of code will add the ripple effect to the RecyclerView elements.
	\begin{enumerate}
		\item  android:clickable="true"
		\item android:focusable="true"
		\item android:foreground="?android:attr/selectableItemBackground"
	\end{enumerate}
	\item LayoutManager - Helps positioning the items
	\item ItemAnimator - Helps with animating the items for common operations such as Addition or Removal of item
\end{enumerate}


\section{Components and Workflow}

\begin{figure}
	\includegraphics[width=\textwidth]{images/recycler/components.png}
	\caption{ RecyclerView is a major enhancement over ListView. It contains many new features like ViewHolder, ItemDecorator, LayoutManager, and SmoothScroller. But one thing that certainly gives it an edge over the ListView is; the ability to have animations while adding or removing an item. }
	\label{fig:recyclercomponents}
\end{figure}


\subsection{RecyclerView.Adapter}
RecyclerView uses an adapter to help convert our model data
into visual representations. A RecyclerView.Adapter  uses a generic to identify a ViewHolder that will be responsible for doing the work to actually tie
model data to row widgets.

RecyclerView.Adapter has three abstract methods that need to be implemented.

\begin{enumerate}
	\item getItemCount() , which fills the same role as does getCount() indicating how many items there will be in the RecyclerView
	\item onCreateViewHolder()  needs to create, configure,
	a ViewHolder for a particular row of our list. It is passed two parameters:
	\begin{enumerate}
		\item a ViewGroup that will hold the views managed by the holder, mostly for use
		with layout inflation, and
		\item an int that is the particular view type we are using, for cases where we have
		multiple view types
	\end{enumerate}
	\item onBindViewHolder()
	is responsible for updating a ViewHolder based upon the
	model data for a certain position .
\end{enumerate}

\subsection{ViewHolder}
The RecyclerView.ViewHolder is responsible for binding data as needed from our
model into the widgets for a row in our list. However, other than needing to use the base class of RecyclerView.ViewHolder ,
there is no other particular protocol that is mandated between the adapter and the
view holder. You can invent your own API.

This solution avoids all the findViewById() method calls in the adapter to find the views to be filled with data.

\subsection{LayoutManagers}
After adding a recyclerview to the activity, the first thing we do is call setLayoutManager() ,
which will associate a RecyclerView.LayoutManager with our RecyclerView, for example a LinearLayoutManager. RecyclerView
knows absolutely nothing about how to lay out its children. That work is delegated
to a RecyclerView.LayoutManager , so that different approaches can be plugged in as
needed.

There are three concrete subclasses of the abstract RecyclerView.LayoutManager
base class that ship with recyclerview-v7 :
\begin{enumerate}
	\item LinearLayoutManager , which implements a vertically-scrolling list
	\item GridLayoutManager ,
	which implements a two-dimensional vertically-
	scrolling list
	\item StaggeredGridLayoutManager , which implements a “staggered grid”, which
	has columns of cells like a GridView , but where the cells do not have to all
	have the same size.
\end{enumerate}


\subsection{ItemAnimator}
RecyclerView.ItemAnimator is a class that defines the animations performed on items and will animate ViewGroup changes such as add/delete/select notified to the adapter. DefaultItemAnimator is a basic animation available by default with the RecyclerView.

To customize the DefaultItemAnimator add an item animator to the RecyclerView. 


\subsection{Responding to Clicks}
Even though displaying elements in RecyclerView is better, in terms of performance, than its predecessors, ListView and GridView, it is not simple to add a clicklistener. 

To overcome this problem create an Interface

\begin{android}
public interface RecyclerViewItemClickListener {
	public void onClick(View view, int position);
	
	public void onLongClick(View view, int position);
}
\end{android}
To detect the item of the RecyclerView which is clicked we need a helper class.

\begin{android}
public class CustomRVItemTouchListener implements RecyclerView.OnItemTouchListener {
	
	//GestureDetector to intercept touch events
	GestureDetector gestureDetector;
	private RecyclerViewItemClickListener clickListener;
	
	public CustomRVItemTouchListener(Context context, final RecyclerView recyclerView, final RecyclerViewItemClickListener clickListener) {
		this.clickListener = clickListener;
		gestureDetector = new GestureDetector(context, new GestureDetector.SimpleOnGestureListener() {
			
			@Override
			public boolean onSingleTapUp(MotionEvent e) {
				return true;
			}
			
			@Override
			public void onLongPress(MotionEvent e) {
				//find the long pressed view
				View child = recyclerView.findChildViewUnder(e.getX(), e.getY());
				if (child != null && clickListener != null) {
					clickListener.onLongClick(child, recyclerView.getChildLayoutPosition(child));
				}
			}
		});
	}
	
	@Override
	public boolean onInterceptTouchEvent(RecyclerView recyclerView, MotionEvent e) {
		
		View child = recyclerView.findChildViewUnder(e.getX(), e.getY());
		if (child != null && clickListener != null && gestureDetector.onTouchEvent(e)) {
			clickListener.onClick(child, recyclerView.getChildLayoutPosition(child));
		}
		return false;
	}
	
	@Override
	public void onTouchEvent(RecyclerView rv, MotionEvent e) {
		
	}
	
	@Override
	public void onRequestDisallowInterceptTouchEvent(boolean disallowIntercept) {
		
	}
}
\end{android}

Basically what this class does is, detect the RecyclerView element under the (X, Y) position where the screen was clicked. This class is helpful for both click types created by the interface.


\section{CardView}
Cards are a popular visual metaphor in mobile development. Dividing content collections (or aspects of a larger piece of content) into cards makes it clearer how you can reorganize that content to fit various screen sizes and orientations. In some cases, you might have a single column of cards, while in other cases, you have cards
arranged more laterally.

In 2014, Google released cardview-v7 , another library in the Android Support
package, that offers a CardView . CardView is a simple subclass of FrameLayout ,
designed to provide a card UI, consisting of a rounded rectangle and a drop shadow.
In particular, CardView will use Android 5.0’s default drop shadows based on widget
elevation, while offering emulated drop shadows on earlier Android releases. This
way, you can get a reasonably consistent look going back to API Level 7.
To use this, you will have to add the cardview-v7 library to your app project.
Android Studio users can just add a dependency on the cardview-v7 artifact in the
Android Support repository

\section{Example}

\begin{example}
	In the repository found here \cite{Buysse2017}, we find an example demonstrating the principles explained above. Moreover, it makes use of a Fragment with a Recyclerview. It uses Butterknife as already demonstrated in previous example. For loading the images it makes use of the Picasso library (see \cite{Square2017}). 
\end{example}

\newpage
\section{Exercise}
\begin{exercise}
	Interview some of your classmates: ask their traits which describe them best. After the interview, make a nice picture (after asking permission of course).
	
	With this information in place, you should implement the following game: Guess who! If you don't know the game, see \cite{WikiHow2017}.
	
	The game should be able to be played in portrait and in landscape mode.
	Portrait mode should only contain a recyclerview with a small description of the student. The student should be able to be swiped out (i.e. he does not meet de traits). Pressing a student should show a new fragment with the student's description.
	Landscape mode should contain the list plus the description of the student. Swiping again deletes the student and the corresponding detailfragment.
	You should be able to restart the game.
	Make sure you apply the give best practices and try to animate as much as possible!
\end{exercise}


\chapterimage{images/persistency/persistence.jpg}

\chapter{Persistence}
In this Chapter we will review some of the persistence methods use in Android. We will have a look into saving a small portion of data and a more structured (relational) type of data.

\section{Shared preferences}
If you have a relatively small collection of key-values that you'd like to save, you should use the SharedPreferences APIs. A SharedPreferences object points to a file containing key-value pairs and provides simple methods to read and write them. Each SharedPreferences file is managed by the framework and can be private or shared.

\subsection{Creating shared preferences}
Using the SharedPreferences class, you can create named maps of name/value pairs that can be persisted across sessions and shared among application components running within the same application sandbox. To create or modify a Shared Preference, call getSharedPreferences on the current Context, passing in the name of the Shared Preference to change.

Shared Preferences are stored within the application’s sandbox, so they can be shared between an application’s components but aren’t available to other applications. To modify a Shared Preference, use the SharedPreferences.Editor class. Get the Editor object by calling edit on the Shared Preferences object you want to change.

To save edits, call apply on the Editor object to save the changes asynchronously.
\begin{android}
SharedPreferences mySharedPreferences = getSharedPreferences(MY_PREFS, 
Activity.MODE_PRIVATE);
SharedPreferences.Editor editor = mySharedPreferences.edit();
// Store new primitive types in the shared preferences object.
editor.putBoolean("isTrue", true);
editor.putFloat("lastFloat", 1f);
editor.putInt("wholeNumber", 2);
editor.putLong("aNumber", 3l);
editor.putString("textEntryValue", "Not Empty");
editor.apply();
\end{android}


\subsection{Retrieving shared preferences}
Accessing Shared Preferences, like editing and saving them, is done using the getSharedPreferences method. Use the type-safe get methods to extract saved values. Each getter takes a key and a default value (used when no value has yet been saved for that key.)

\begin{android}
// Retrieve the saved values.
boolean isTrue = mySharedPreferences.getBoolean("isTrue", false);
float lastFloat = mySharedPreferences.getFloat("lastFloat", 0f);
int wholeNumber = mySharedPreferences.getInt("wholeNumber", 1);
long aNumber = mySharedPreferences.getLong("aNumber", 0);
String stringPreference =
mySharedPreferences.getString("textEntryValue", "");
\end{android}

\section{Room}
Android provides a built-in SQLite database for when the data you want to store is more complex than just key-value pairs.
Oftentimes this is used to provide caching and/or offline functionality: once a network request has been made the result is also saved in the applications database so it can be accessed when no network is available.

Room, just like ViewModel and LiveData, is one of the recently released Architecture Components.
It's a wrapper around the built-in SQLite functionalities.
Working directly with the SQLite API is possible, but it is very low-level and requires a great deal of time and effort to use: \cite{Sqlite}
\begin{enumerate}
	\item There is no compile-time verification of raw SQL queries.
	As your data graph changes, you need to update the affected SQL queries manually.
	This process can be time consuming and error prone.
	\item You need to use lots of boilerplate code to convert between SQL queries and data objects.
\end{enumerate}

\subsection{In theory}
There are 3 major components in Room \cite{RoomDevAndroid} (see also figure \ref{fig:roomarchitecture}):
\begin{description}
	\item[Database]: Contains the database holder and serves as the main access point for the underlying connection to your app's persisted, relational data.
	The class that's annotated with @Database should satisfy the following conditions:
	\begin{enumerate}
		\item Be an abstract class that extends RoomDatabase.
		Include the list of entities associated with the database within the annotation.
		\item Contain an abstract method that has 0 arguments and returns the class that is annotated with @Dao.
	\end{enumerate}
 	At runtime, you can acquire an instance of Database by calling Room.databaseBuilder() or Room.inMemoryDatabaseBuilder().
	\item [DAO]: Contains the methods used for accessing the database.
	\item [Entity]: Represents a table within the database.
\end{description}

\begin{figure}
	\centering
	\includegraphics[width=0.7\linewidth]{images/persistency/room_architecture}
	\caption{Room architecture diagram. From \cite{RoomDevAndroid}.}
	\label{fig:roomarchitecture}
\end{figure}

\subsection{In practice}
In this section we'll follow a slightly adapter version of Google's Room with a View Codelab \cite{RoomWithAView}.
We won't be using Kotlin's co-routine feature and will also uses multiple fragments instead of multiple activities. 


\subsection{Setting up the gradle files}
After starting a new project with just an empty Activity, make the following changes to the app's build.gradle file.

Add the Kotlin annotation processor.
This processor will read the annotations we put with the Database, Entity and DAO classes.
\begin{android}
	apply plugin: 'kotlin-kapt'
\end{android}

And add the necessary libraries:
\begin{android}
// Room components
implementation "android.arch.persistence.room:runtime:1.1.1"
kapt "android.arch.persistence.room:compiler:1.1.1"
androidTestImplementation "android.arch.persistence.room:testing:1.1.1"

// Lifecycle components
implementation "android.arch.lifecycle:extensions:1.1.1"
kapt "android.arch.lifecycle:compiler:1.1.1"
\end{android}

\subsection{Creating the entity}
Every class you want to save in the database has to become a database \textit{entity}.
In this simple application we will allow the user to add words to a database.

This means creating a Word class and giving it the required annotations.
\begin{description}
	\item[@Entity] is used to indicate to the Room library that objects of this class will be stored in the database.
	The annotation also requires a tableName parameter that specifies the name of the table in which the objects will be stored.
	\item[@PrimaryKey] indicates, as you would expect, which attribute is the primary key for this table.
	\item[@ColumnInfo] can be used if you want the column name to be different from the name of the attribute.
\end{description}

\begin{android}
@Entity(tableName = "word_table")
class Word(@PrimaryKey @ColumnInfo(name = "word") val word: String)
\end{android}

\subsection{Creating the DAO}
A DAO is a \textbf{D}ata \textbf{A}ccess \textbf{O}bject.
With this object Room allows us to write the required SQL queries. 
Just like in Retrofit we define an interface (an abstract class is possible as well) with the methods we'd like to call and annotate each method with the corresponding SQL query.

Room provides some predefined annotations for the most common queries (@Insert, @Update, @Delete).
For other queries the @Query annotation should be used, with the actual query as parameter for the annotation.

\begin{android}
/**
* Use the @Query annotation to specify a custom SQL query.
* By specifiying LiveData as the return value Room will provide all the necessary code to update the LiveData object
* when the database is updated.
*/
@Query("SELECT * from word_table ORDER BY word ASC")
fun getAllWords(): LiveData<List<Word>>


@Insert
fun insert(word: Word)

@Query("DELETE FROM word_table")
fun deleteAll()
\end{android}

\subsection{Adding a database}
The DAO itself isn't enough to perform the queries.
Just like a Retrofit API needs a backend to respond to the requests, we need a database to query. 

The database class should be abstract and extend RoomDatabase.
The @Database annotation is used to indicate what entities the database will hold, as well as indicate a version number
\footnote{This version number can be upped each time you make changes to the database scheme. 
	If you want to migrate the data from one schema version to the next you'll have to write Migration classes and include them when building the database.
	More info on \url{https://medium.com/androiddevelopers/understanding-migrations-with-room-f01e04b07929}.}
Every DAO that will be used to perform queries on this database should be mentioned as return values for abstract functions.
When building the database these interfaces will be implemented for you.

The database class should also be implemented as a singleton to avoid multiple instances of your database being opened at the same time.

\begin{android}
@Database(entities = [Word::class], version = 1)
abstract class WordDatabase : RoomDatabase() {
	abstract fun wordDao(): WordDao
	
	companion object {
		@Volatile
		private var INSTANCE: WordDatabase? = null
		
		fun getDatabase(context: Context): WordDatabase {
			val tempInstance = INSTANCE
			if (tempInstance != null) {
				return tempInstance
			}
			synchronized(this) {
				val instance = Room.databaseBuilder(
					context.applicationContext,
					WordDatabase::class.java,
					"Word_database"
				).build()
				INSTANCE = instance
				return instance
			}
		}
	}
}
\end{android}

\subsection{The repository pattern}
You've known about repositories since your first year: they're classes that represent a collection of objects and the ones that are responsible for communication with the persistence layer of you application. 
This latter is what we'll be doing here: the repository class will abstract the access to multiple data sources (network and database).
A simple diagram is shown in figure \ref{fig:repository}.

\begin{figure}
	\centering
	\includegraphics[width=0.7\linewidth]{images/persistency/repository}
	\caption{A Repository class abstracts access to multiple data sources.
		When a network connection is unavailable, the Dao can be used to request cached data instead.
		Using the repository pattern hides this complexity from other classes.}
	\label{fig:repository}
\end{figure}

The repository usually implements the logic for deciding whether to fetch data from a network or use results cached in a local database.
In this simple example we'll only implement the database portion. 

For every operation a method should be implemented. 
These methods will call the corresponding DAO.
Database calls should always be done asynchronously to avoid blocking the main thread.
To enforce this we add the @WorkerThread annotation to each method.

\begin{android}
class WordRepository(private val wordDao: WordDao) {
	val words: LiveData<List<Word>> = wordDao.getAllWords()
	
	@WorkerThread
	fun insert(word: Word) {
		wordDao.insert(word)
	}
}
\end{android}


\subsection{Using the DAO}
We will use a ViewModel to hold the list of all words that have been entered.
This ViewModel will have to load the list from the database when it is created, and provide a function that inserts a given word into the database.
To do this it needs a WordRepository.

\begin{android}
class WordViewModel : ViewModel() {
	@Inject
	lateinit var wordRepository: WordRepository
	
	init {
		App.component.inject(this)
	}
	
	val allWords: LiveData<List<Word>> = wordRepository.words
	
	fun insert(word: Word) {
		doAsync {
			wordRepository.insert(word)
		}
	}
}
\end{android}

The wordRepository is injected using Dagger. 
How this is done is slightly different from the way explained in the network chapter:
to create a SQLite database you require a context. 
You can't create a context object yourself, so it has to be given to the Module somehow. 
This is done by creating our own Application class, registering it in the manifest file, and passing itself as a constructor parameter for the Module.
The created component is saved in a companion object and can thus be used everywhere throughout our app.

\begin{android}	
class App : Application() {
	companion object {
		lateinit var component: DatabaseComponent
	}
	
	override fun onCreate() {
		super.onCreate()
		component = DaggerDatabaseComponent
						.builder()
						.databaseModule(DatabaseModule(this))
						.build()
	}
}
\end{android}

This ViewModel is now ready to be used.
In a MainFragment we create the ViewModel and ask it for the list of words so we can fill the RecyclerView.
In a NewWordFragment we allow the user to enter a word and ask the ViewModel to save it (to the database).

\chapterimage{images/networking/network}

\chapter{Network}
This chapter will explain how to combine all previous (Android) concepts: MVVM with Kotlin, Android Architecture Components and incorporating Dagger 2, Retrofit and RxAndroid to be able to create network requests. 

If you are not completely up to date with the previous chapters, we recommend that you take a closer look to those chapters before continuing. 

The example application is able to download METAR information from different airports, which is a code indicating the type of weather at that airport \cite{Wikipedia2018}.
The application parses this information to make it more readable for non-pilots. 

Parts of the application are inspired by the following blog posts: \cite{Gahfy2018, Tiwari2018, Saquib2018}

The complete build.gradle files are displayed here. 

\lstinputlisting[language=Kotlin, 
caption={Top-level build file for the project},
label=code:netwgradle]{srccode/network/build.gradle}


\lstinputlisting[language=Kotlin, 
caption={Module build file for the project},
label=code:netwgradle2]{srccode/network/build.gradle2}

\section{Base classes}
First we need to define the models and ViewModels we are going to use to represent a METAR.
Therefore, we have package called \lstinline!model! which contains the model class which is a simple POJO.

\lstinputlisting[language=Kotlin,firstline=32,lastline=42 ,
caption={METAR POJO},
label={code:netwmodel}]{srccode/network/java/be/equality/metar/model/Metar.kt}

There are two remarks here:
\begin{itemize}
	\item All the properties are annotated wit \lstinline!@field:Json(name = "field-name")!.
	This is because we are making use of Moshi, see section \ref{sec:moshi}
	\item  The class implements \lstinline|Parcelable| with the annotation \lstinline|@Parcelize|.
	The Android Extensions plugin  includes an automatic \lstinline|Parcelable| implementation generator.
	By declaring the serialized properties in a primary constructor and add a \lstinline|@Parcelize| annotation, and writeToParcel()/createFromParcel() methods will be created automatically.
\end{itemize}

\section{Retrofit}
Retrofit is a REST Client for Android and Java by Square.
It makes it relatively easy to retrieve and upload JSON (or other structured data) via a REST based webservice.
In Retrofit you configure which converter is used for the data serialization.
Typically for JSON you use GSon, but you can add custom converters to process XML or other protocols.
You can also use Moshi which is a modern JSON library for Android and Java and makes it easy to parse JSON into Java objects.

Before we go into the Retrofit code, you need to set he following permissions. 
\lstinputlisting[language=Kotlin,firstline=5,lastline=5,
caption={Setting the network permissions for using the network},
label=code:manifest]{srccode/network/AndroidManifest.xml}


To work with Retrofit you basically need the following three classes:
\begin{itemize}
	\item Model class which is used as a JSON model. We already have this, see code listing \ref{code:netwmodel}
	\item Interfaces that define the possible HTTP operations, see listing \ref{code:netwApi}.
	Each call from the created service can make a synchronous or asynchronous HTTP request to a remote webserver.
	\item Retrofit.Builder class. An Instance which uses the interface and the Builder API to allow defining the URL end point for the HTTP operations.
\end{itemize}

\lstinputlisting[language=Kotlin,firstline=1,lastline=19 
caption={Network service API},
label=code:netwApi]{srccode/network/java/be/equality/metar/network/MetarApi.kt}


To generate a builder, see the following code snippet.

\lstinputlisting[language=Kotlin,firstline=53,lastline=59,
caption={Using a builder to create the Network Service API},
label=code:netwApi]{srccode/network/java/be/equality/metar/injection/module/NetworkModule.kt}



\subsection{Moshi}
\label{sec:moshi}
Moshi is a modern JSON library for Android and Java from Square. It can be considered as the successor to GSON, with a simpler and leaner API and an architecture enabling better performance while also being the most Kotlin-friendly library you can use to parse JSON files, as it comes with Kotlin-aware extensions.

We refer to the documentation of this library to use it. In our example we have used the \lstinline!@field:Json(name = "field-name")! annotation to map the properties to the JSON variables. 


\section{Dagger2}
When working with libraries such as Retrofit, you will notice that your code will have a lot of dependencies (e.g. Moshi). Remeber the course Analyse 1?: 

\begin{framed}
	Whenever a class A uses another class or interface B, then A depends on B. A cannot carry out it's work without B, and A cannot be reused without also reusing B. In such a situation the class A is called the "dependant" and the class or interface B is called the "dependency".
\end{framed}
Dependencies are bad because they decrease reuse and make testing more difficult. One way of overcoming the dependency problem is \texttt{Dependency injection}. It is a programming technique that makes a class independent of its dependencies. It achieves that by decoupling the usage of an object from its creation. This helps you to follow SOLID's dependency inversion (see section \ref{sec:dip}) and single responsibility principles (see section )\ref{sec:srp}).

There are few annotations in Dagger 2 API. We will use some of them and explain them along the way. 
A Module defines one or more injectable classes (as denoted by the Provides annotation). In our case we would like the Retrofit interface to be able to be injected in our other classes. This way we only have one anchor point to our network service.  Therefore  create a module to inject the Retrofit instance, called NetworkModule.

\lstinputlisting[language=Kotlin,firstline=1,lastline=61,
caption={The Dagger module used to inject the Network Service into the classes of your project. },
label=code:netwApi]{srccode/network/java/be/equality/metar/injection/module/NetworkModule.kt}


The next Dagger 2 concept is a Component. A Component is a mapping between one or more modules and one or more classes that will use them. In this particular case, we have the NetworkModule which needs to inject dependencies in our ViewModel. We have not yet defined our ViewModel, but we will in the next section. 

Components can be instantiated by using the Builders generated by Dagger, but  allows us to customize the generated builder by something knows as a Component.Builder. 

\lstinputlisting[language=Kotlin,firstline=1,lastline=35),
caption={Using a component to link the modules with the injecting classes.},
label=code:netwApi]{srccode/network/java/be/equality/metar/injection/component/ViewModelInjectorComponent.kt}


\section{MVVM}
The idea of MVVM has already been introduced in chapter \ref{cap:mvvm}, so we will only focus on the new topics introcuded in this chapter. 

Let us create a \lstinline!MetarViewModel! class which will be our ViewModel. For now, we will only get results from API then display it in the view.

First thing we will need so is an instance of the MetarApi class in order to get the result from the API. This instance will be injected by Dagger by using the \lstinline!@Inject! anotation.

\lstinputlisting[language=Kotlin,firstline=14,lastline=26),
caption={Injecting into the ViewModel},
label=code:netwApi]{srccode/network/java/be/equality/metar/ui/MetarViewModel.kt}


We will also create a BaseViewModel class which will inject the correct dependencies, depending on the subtype of this subclass. We can later use this way for other injections which could be necessary.

\lstinputlisting[language=Kotlin,firstline=1,lastline=34),
caption={BaseViewModel class},
label=code:netwApi]{srccode/network/java/be/equality/metar/base/BaseViewModel.kt}



And there you have it: the correct dependencies are injected in our ViewModel, without the use of any constructor. 

Don't forget to update the Fragment for the ViewModel and the Layout files. 


\section{ReactiveX/RxAndroid}
Now that the injection of the MetarAPI has been done, let’s retrieve the data from the API. We will need to perform the call in background thread while we want to perform actions with the result on Android main thread. 

\begin{framed}
	To avoid creating an unresponsive UI, don't perform network operations on the UI thread. By default, Android 3.0 (API level 11) and higher requires you to perform network operations on a thread other than the main UI thread; if you don't, a NetworkOnMainThreadException is thrown.
\end{framed}

RxAndroid adds the minimum classes to RxJava that make writing reactive components in Android applications easy and hassle-free. More specifically, it provides a Scheduler that schedules on the main thread or any given Looper. 

\subsection{Reaxctive Programming}
Writing Reactive code has gained a lot of attention the last months. Reactive programming is programming aimed at flows.
The main idea is in presenting events and data as flows that can be unified, filtered, transformed, and separated.
The basic building blocks of reactive code are Observables and Subscribers. The Observable class is the source of data and the Subscriber class is the consumer.

\subsection{Observable}
One of the main classes in RxJava is Observable<T>.
Its public interface has the following methods: onNext(), onComplete(), onError().
The Observable class has two states: the complete successfully state, after which Observable stops working and the error stoppage state.
Observable regulates when to supply data. We are supposed to react to those supplies. This class provides that flow of data and events.

It is necessary to remember to unsubscribe from the asynchronous calls.
Rx allows you to conveniently unsubscribe from Observable.
When subscribing to an Observable, the subscribe method returns the Subscription object that contains the unsubscribe() method.
In other words, this is some sort of a conversion chain.
When calling unsubscribe(), all the operators unsubscribe from one another in sequence from top to bottom.
This is how you can avoid memory leaks.

\subsection{subscribeOn}
With the help of subscribeOn, we specify the background thread where the data flow will be generated and processed.
subscribeOn accepts Scheduler as a parameter.
A Scheduler is an abstraction for the thread pool management, just like ExecutorService in Java.
In RxJava there are several off-the-shelf implementations of Scheduler and we will use Schedulers.io().
Perfect for the I/O activities, like reading or recording into a database, server requests, reading of or recording to persistent storage.
In other words, the operations that require complex computations and waiting for data to be sent or received.

\subsection{observeOn}
observeOn is supposed to redirect the chain of operations to another thread.
As opposed to subscribeOn, the observeOn position influences the way operations after it in the chain will be processed..

We use two more operators, to show or hide the loading state. 
\begin{itemize}
	\item doOnSubscribe which modifies the source so that it invokes the given action when it is subscribed from its subscribers.
	\item  doOnTerminate which calls the specified action just before this Observable signals onError or onCompleted.
\end{itemize}

In the end, we get the following code:

\lstinputlisting[language=Kotlin,firstline=34,lastline=88),
caption={Using RxAndroid to schedule the requests on another thread and applying other actions.},
label=code:netwApi]{srccode/network/java/be/equality/metar/ui/MetarViewModel.kt}





\chapterimage{images/memory/leaks}

\chapter{Memory Leaks}


\section{Garbage Collection in Android}
Random-access memory (RAM) is a valuable resource in any software development environment, but it's even more valuable on a mobile operating system where physical memory is often constrained. Although both the Android Runtime (ART) and Dalvik virtual machine perform routine garbage collection, this does not mean you can ignore when and where your app allocates and releases memory. You still need to avoid introducing memory leaks, usually caused by holding onto object references in static member variables, and release any reference objects at the appropriate time as defined by lifecycle callbacks.

In this section we will focus on memory management in ART.

\section{Memory Basics}

\subsection{RAM}
RAM, which is short for random access memory, is one of the critical components of the smartphone along with the processing cores and dedicated graphics. Without RAM in any sort of computing system like this your smartphone would fail to perform basic tasks because accessing files would be ridiculously slow.

This type of memory is a middle man between the file-system, which is stored on the ROM, and the processing cores, serving any sort of information as quickly as possible. Critical files that are needed by the processor are stored in the RAM, waiting to be accessed. These files could be things such as operating system components, application data and game graphics; or generally anything that needs to be accessed at speeds faster than other storage can provide.

\subsection{ROM}
Like RAM, internal storage is critical to a smartphone’s operation; without any place to store the operating system and critical files there would be nothing for the phone to do. Even if a phone has no storage accessible to the user, there will also be some form of internal storage that stores the operating system.

Depending on the operating system loaded on the device, and the device itself, there are multiple storage chips inside the device. These chips may then be partitioned into several areas for different purposes, such as application storage, cache and system files. Normally the chip that stores the system files is called the ROM for read-only memory; however this is a bit of a misnomer as the memory here can actually be modified through system updates, just not by the end user.

\subsection{Stack Memory}
Stack memory in Android is used to store datatypes, function calls and reference variables for objects (the handle to objects).
Its contains short-lived (weak) references for objects stored in the heap memory. Whenever a function is called, a block of memory is reserved for that function and its local primitive values and reference to other objects stored inside that functions. When the function execution ends, the function block space will be free and available for other methods.

\subsection{Heap Memory}
Heap memory  in Android is used for storing Objects and java classes. When we  create an object it will take space in heap memory. Android’s memory heap is a generational one, meaning that there are different buckets of allocations that it tracks, based on the expected life and size of an object being allocated. For example, recently allocated objects belong in the young generation. When an object stays active long enough, it can be promoted to an older generation, followed by a permanent generation.

Each heap generation has its own dedicated upper limit on the amount of memory that objects there can occupy. Any time a generation starts to fill up, the system executes a garbage collection event in an attempt to free up memory. The duration of the garbage collection depends on which generation of objects it's collecting and how many active objects are in each generation.

\subsection{Shared versus private memory}
There are common framework classes, assets and native libraries that are utilized by all apps. TO save memory, Android uses shared memory for those assets.

Private memory is memory thas is used just by your app and not used by other apps. 

\subsubsection{Zygote}
Zygote \footnote{The zygote's genome is a combination of the DNA in each gamete, and contains all of the genetic information necessary to form a new individual. In multicellular organisms, the zygote is the earliest developmental stage.} is the method Android uses to start apps. Rather than having to start each new process from scratch and loading the whole system and the Android framework afresh each time you want to start an app, it does that process only once, and then stops at that point. In this initialization nothing app-specific happens. Then, when you want to start an app, the Zygote process forks, and the child process continues where it left off, loading the app itself into the VM.

\subsection{Dirty Memory vs. Clean Memory}
DIrty memory nis memory that is only stored in the RAM. Clean memory consists of items in RAM that are also saved on the disk. 

One of the main feature of the ART runtime is that apps are compiled on install. On devices running ART the app code is compiled at install and ready on disk. Because now the app code in memory is now by definition clean, memory management in ART is improved.

\section{Garbage collection}
garbage collection (GC) is a form of automatic memory management. The garbage collector, or just collector, attempts to reclaim garbage, or memory occupied by objects that are no longer in use by the program. 

The basic principles of garbage collection are to find data objects in a program that cannot be accessed in the future, and to reclaim the resources used by those objects.

The overall strategy consists of determining which objects should be garbage collected by tracing which objects are reachable by a chain of references from certain root objects, and considering the rest as garbage and collecting them.

ART has a few different GC plans that consist of running different garbage collectors. The default plan is the CMS (concurrent mark sweep). Mainly ART uses a  non-moving generational garbage collector. It scans only the portion of the heap that was modified since the last GC and can reclaim only the objects allocated since the last GC. In addition to the CMS plan, ART performs heap compaction when an app changes process state to a jank-imperceptible process state (e.g. background or cached). The goal of this compaction is to reduce memory usage of backgrounded apps. Currently, the event that triggers heap compaction is ActivityManager process-state changes. When an app goes to background, it notifies ART the process state is no longer jank “perceptible.” This enables ART do things that cause long application thread pauses, such as compaction and monitor deflation. 

\section{How much memory is used by an App}
\subsection{adb shell dumpsys meminfo}
The biggest consumers of memroy are bitmaps. No matter how well you have compresed the file or network transmission, PNG and JPEG files use 32 bits per pixe, meaning that an 100 x 100 pixel thumbnail can use 320.000 bits of memory.

ActivityManager.getMemoryClass will return the maximum size for you app' heap. Depending on this value, you can scale your application. 

To find out how much memory an app is using you can use \texttt{adb shell dumpsys meminfo}. 

\begin{verbatim}
eothein@eothein-Alienware-13:~/Android/Sdk/platform-tools$ adb shell dumpsys meminfo
adb server is out of date.  killing...
* daemon started successfully *
Applications Memory Usage (in Kilobytes):
Uptime: 60002338 Realtime: 170929502

Total PSS by process:
258,425K: system (pid 3763)
191,773K: com.facebook.orca (pid 21195 / activities)
162,638K: com.podcast.podcasts (pid 1513 / activities)
139,611K: com.android.systemui (pid 4144 / activities)
136,684K: com.google.android.gms (pid 29546)
112,756K: com.facebook.katana (pid 17090 / activities)
106,035K: com.sec.android.inputmethod (pid 7578)
104,495K: com.samsung.android.bixby.agent (pid 8652)
100,113K: com.lastpass.lpandroid (pid 22401 / activities)
97,899K: com.twitter.android (pid 2244)
87,537K: com.sec.android.app.launcher (pid 5387 / activities)
78,493K: com.google.android.gms.persistent (pid 5172)
77,809K: be.hogent.jensbuysse.metartaff (pid 1059 / activities)

\end{verbatim}

Proportional Set Size (PSS) is the total memory used by the app, which is the private memory together with the shared memory.

You can also add the PID number to the meminfo command.
\begin{verbatim}
eothein@eothein-Alienware-13:~/Android/Sdk/platform-tools$ adb shell dumpsys meminfo 1059
Applications Memory Usage (in Kilobytes):
Uptime: 60276648 Realtime: 171203812

** MEMINFO in pid 1059 [be.hogent.jensbuysse.metartaff] **
Pss  Private  Private  SwapPss     Heap     Heap     Heap
Total    Dirty    Clean    Dirty     Size    Alloc     Free
------   ------   ------   ------   ------   ------   ------
Native Heap     4552     4540        0     5626    22528    16537     5990
Dalvik Heap     4230     4208        0    10399    31791    19075    12716
Dalvik Other     1025     1024        0      712                           
Stack      272      272        0      136                           
Ashmem        4        4        0        0                           
Other dev        4        0        4        0                           
.so mmap     2539      120      360     2516                           
.apk mmap      347        0       64        0                           
.ttf mmap      110        0       36        0                           
.dex mmap     1392     1328       64     3016                           
.oat mmap     2365        0      328        0                           
.art mmap     1067      752        0      378                           
Other mmap       19        4        4        0                           
EGL mtrack    28368    28368        0        0                           
Unknown      124      124        0      408                           
TOTAL    46418    40744      860    23191    54319    35612    18706

App Summary
Pss(KB)
------
Java Heap:     4960
Native Heap:     4540
Code:     2300
Stack:      272
Graphics:    28368
Private Other:     1164
System:    28005

TOTAL:    46418       TOTAL SWAP PSS:    23191

Objects
Views:       42         ViewRootImpl:        1
AppContexts:        3           Activities:        1
Assets:       14        AssetManagers:        2
Local Binders:       12        Proxy Binders:       24
Parcel memory:        4         Parcel count:       16
Death Recipients:        0      OpenSSL Sockets:        0

SQL
MEMORY_USED:        0
PAGECACHE_OVERFLOW:        0          MALLOC_SIZE:        0

\end{verbatim} 

This is the memory usage of the app while it is in the foreground. This way we can break down how the memory allocation is distributed over the components of our application. 

The second table is showing information about things that are using mempory: view count, asset count and the number of activities. If for some reason these numbers are high, or by rerunning the command get bigger you likely have a memory issue to investigate. 

\subsection{adb shell dumpsys procstats}
Android 4.4 KitKat introduced a new system service called procstats that helps you better understand how your app is using the RAM resources on a device. Procstats makes it possible to see how your app is behaving over time — including how long it runs in the background and how much memory it uses during that time. It helps you quickly find inefficiencies and misbehaviors in your app that can affect how it performs, especially when running on low-RAM devices. In the following example we see the memory usage for the last three hours.

\begin{verbatim}
adb shell dumpsys procstats --hours 3 be.hogent.jensbuysse.metartaff
AGGREGATED OVER LAST 3 HOURS:
System memory usage:
SOff/Norm: 4 samples:
Cached: 99MB min, 146MB avg, 184MB max
Free: 20MB min, 29MB avg, 41MB max
ZRam: 0,00 min, 0,00 avg, 0,00 max
Kernel: 721MB min, 731MB avg, 741MB max
Native: 194MB min, 208MB avg, 231MB max
SOn /Norm: 32 samples:
Cached: 94MB min, 156MB avg, 304MB max
Free: 12MB min, 31MB avg, 62MB max
ZRam: 0,00 min, 0,00 avg, 0,00 max
Kernel: 706MB min, 754MB avg, 779MB max
Native: 34MB min, 70MB avg, 219MB max

Per-Package Stats:
* be.hogent.jensbuysse.metartaff / u0a307 / v1:
* be.hogent.jensbuysse.metartaff / u0a307 / v1:
TOTAL: 5,3% (68MB-55MB-80MB/41MB-60MB-75MB over 4)
Top: 5,3% (68MB-55MB-80MB/41MB-60MB-75MB over 4)
Bg TOTAL: 4,3% (38MB-38MB-38MB/9,0MB-9,0MB-9,0MB over 1)
(Last Act): 1,7% (38MB-38MB-38MB/9,0MB-9,0MB-9,0MB over 1)
(Cached): 2,6%

Summary:
* be.hogent.jensbuysse.metartaff / u0a307 / v1:
TOTAL: 5,3% (68MB-55MB-80MB/41MB-60MB-75MB over 4)
Top: 5,3% (68MB-55MB-80MB/41MB-60MB-75MB over 4)
Bg TOTAL: 4,3% (38MB-38MB-38MB/9,0MB-9,0MB-9,0MB over 1)
(Last Act): 1,7% (38MB-38MB-38MB/9,0MB-9,0MB-9,0MB over 1)
(Cached): 2,6%

Run time Stats:
SOff/Norm: +1h57m9s335ms
SOn /Norm: +32m35s716ms
TOTAL: +2h29m45s51ms

Memory usage:
Kernel : 736MB (252 samples)
Native : 178MB (252 samples)
Persist: 251MB (198 samples)
Top    : 116MB (816 samples)
ImpFg  : 401MB (1872 samples)
ImpBg  : 420KB (578 samples)
Service: 230MB (1610 samples)
Receivr: 299KB (1647 samples)
Home   : 24MB (88 samples)
LastAct: 13MB (776 samples)
CchAct : 29MB (72 samples)
CchCAct: 40MB (81 samples)
CchEmty: 342MB (1075 samples)
Cached : 148MB (252 samples)
Free   : 30MB (252 samples)
TOTAL  : 2,5GB
ServRst: 2,0MB (746 samples)

Start time: 2017-11-15 07:38:00
Total elapsed time: +7h6m26s29ms (partial) (swapped-out-pss) libart.so

Available pages by page size:
Zone   0       Unmovable      0    81     5     0     0     0     0     0     0     0     0
Zone   0         Movable   4826  1986   366    14     0     0     0     0     0     0     0
Zone   0     Reclaimable      1    25     1     0     0     0     0     0     0     0     0
Zone   0      HighAtomic     69    53    25    22    11     5     2     0     0     0     0
Zone   0             CMA      0     0     0     0     0     0     0     0     0     0     0
Zone   0         Isolate     21    15    17    19    15     9     5     3     0     1     0

\end{verbatim}

\section{Tools for tracking memory leaks}
\subsection{Android Profiler}
The Memory Profiler is a component in the Android Profiler that helps you identify memory leaks and memory churn that can lead to stutter, freezes, and even app crashes. It shows a realtime graph of your app's memory use, lets you capture a heap dump, force garbage collections, and track memory allocations.

To open the Memory Profiler, follow these steps:

\begin{enumerate}
	\item Click View > Tool Windows > Android Profiler (you can also click Android Profiler  in the toolbar).
	\item Select the device and app process you want to profile from the Android Profiler toolbar. If you've connected a device over USB but don't see it listed, ensure that you have enabled USB debugging.
	\item Click anywhere in the MEMORY timeline to open the Memory Profiler.
\end{enumerate}

\begin{figure}[h]
	\includegraphics[width=\textwidth]{images/memory/profiler.png}
	\caption{Android provides a managed memory environment . When it determines that your app is no longer using some objects, the garbage collector releases the unused memory back to the heap}
	\label{fig:profiler}
\end{figure}

Explanation can be found \href{https://developer.android.com/studio/profile/memory-profiler.html}{here.}

\subsection{Leak Canary}
LeakCanary is an Open Source Java library to detect memory leaks in your debug builds. It is the canary in the coal mine for memory leaks: it detects memory leaks before any out-of-memory crashed are thrown. 

\section{Memory leaks: common leak patterns}
There are lots of ways you can cause a memory leak in Android. To summarize, there are mainly three categories.

\begin{enumerate}
	\item Leak activity to a static reference
	\item Leak activity to a worker thread
	\item Leak thread itself
\end{enumerate}

\subsection{Leak activity to a static reference}
A static reference lives as long as your app is in memory. An activity has lifecycles which are usually destroyed and re-created multiple times during you app’s lifecycle. If you reference an activity directly or indirectly from a static reference, the activity would not be garbage collected after it is destroyed. An activity can range from a few kilo bytes to many mega bytes depending on what contents are in it. If it has a large view hierarchy or high resolution images, it can make a large chunk of memory leaked.

\subsection{Leack activity to worker thread}
A worker thread can also out-live an Activity. If you reference an Activity directly or indirectly from a worker thread which lives longer than the Activity, you also leak the Activity object.

\subsection{Leak thread}
Every time you start a worker thread from an activity, you are responsible of managing the worker thread yourself. Because the worker thread can live longer than the Activity, you should stop the worker thread properly when the Activity is destroyed. If you forget to do so, you are risking leaking the worker thread itself.

These leaks are illustrated by \href{https://github.com/frank-tan/SinsOfMemoryLeaks}{Frank Tan}.

\newpage
\section{Exercises}
\begin{exercise}
	Add Leak Canary to the DotPict-, WhoIsIt and Metar application and find out if any of them suffer from memory leaks. Also use the profiler and the other tools which are described above. If so, you should do the following things:
	\begin{itemize}
		\item Create a new branch where you fix the problem
		\item Document the memory leak by writing a short paragraph on how the leak got constructed, how you detected it and how you solved it. Illustrate the documentation with the android profiler and abd information.
	\end{itemize}
	
\end{exercise}

\chapterimage{images/firebase/firebase}

\chapter{Firebase}

\section{Introduction}
With Firebase, you can store and sync data to a NoSQL cloud database. The data is stored as JSON, synced to all connected clients in realtime, and available when your app goes offline. It offers API's that enable you to authenticate users with email and password, Facebook, Twitter, GitHub, Google, anonymous auth, or to integrate with existing authentication system. Other than the Realtime Database and Authentication, it offers a myriad of other services including Cloud Messaging, Storage, Hosting, Remote Config, Test Lab, Crash Reporting, Notification, App Indexing, Dynamic Links, Invites, AdWords, AdMob.

In this chapter we will touch on the basic functionalities Firebase has to offer.

We will follow the tutorial provided by \cite{Developpers}.

\section{Some notes on setting up}
Before starting the Android project, head over to firebase.google.com and create an account. After logging in to your account, head over to the Firebase console and create a project that will hold your app’s data.

Make sure you add the correct information when creating the firebase app. 

Certain Google Play services (such as Google Sign-in and App Invites) require you to provide the SHA-1 of your signing certificate so we can create an OAuth2 client and API key for your app. To get the SHA-1:

\begin{enumerate}
	\item Open Android Studio
	\item Open your Project
	\item Click on Gradle (From Right Side Panel, you will see Gradle Bar)
	\item Click on Refresh (Click on Refresh from Gradle Bar, you will see List Gradle scripts of your Project)
	\item Click on Your Project (Your Project Name form List (root))
	\item Click on Tasks
	\item Click on Android
	\item Double Click on signingReport (You will get SHA1 and MD5 in Run Bar (Sometimes it will be in Gradle Console))
\end{enumerate}

Make sure you add the dependencies correctly.




\chapterimage{images/testing/usertesting.jpg}

\chapter{Testing in Android}

\section{The testing pyramid}
The Testing Pyramid, shown in Figure 2, illustrates how your app should include the three categories of tests: small, medium, and large:

\begin{figure}[h]
	\centering
	\includegraphics[width = 0.5\textwidth]{images/testing/pyramid}
	\caption{The Testing Pyramid, showing the three categories of tests that you should include in your app's test suite}
\end{figure}


Small tests are unit tests that you can run in isolation from production systems. They typically mock every major component and should run quickly on your machine.


Medium tests are integration tests that sit in between small tests and large tests. They integrate several components, and they run on emulators or real devices.


Large tests are integration and UI tests that run by completing a UI workflow. They ensure that key end-user tasks work as expected on emulators or real devices.


Although small tests are fast and focused, allowing you to address failures quickly, they're also low-fidelity and self-contained, making it difficult to have confidence that a passing test allows your app to work. You encounter the opposite set of tradeoffs when writing large tests.

Because of the different characteristics of each test category, you should include tests from each layer of the test pyramid. Although the proportion of tests for each category can vary based on your app's use cases, we generally recommend the following split among the categories: 70 percent small, 20 percent medium, and 10 percent large.

\section{Unit tests}
Unit tests are the fundamental tests in your app testing strategy. By creating and running unit tests against your code, you can easily verify that the logic of individual units is correct. Running unit tests after every build helps you to quickly catch and fix software regressions introduced by code changes to your app.

In your Android Studio project, you must store the source files for local unit tests at
\texttt{module-name/src/test/java/}. 
 
 This directory already exists when you create a new project.
 
 Your local unit test class should be written as a JUnit 4 test class. JUnit is the most popular and widely-used unit testing framework for Java. 

You also need to configure the testing dependencies for your project to use the standard APIs provided by the JUnit 4 framework.

To create a basic JUnit 4 test class, create a Java class that contains one or more test methods. A test method begins with the @Test annotation and contains the code to exercise and verify a single functionality in the component that you want to test.

See the 2048 example where a Unit Test has been added for the business logic.

\section{Espresso}
Espresso automatically synchronizes your test actions with the user interface of your application. The framework also ensures that your activity is started before the tests run. It also let the test wait until all observed background activities have finished.

It is intended to test a single application but can also be used to test across applications. If used for testing outside your application, you can only perform black box testing, as you cannot access the classes outside of your application.

Espresso has basically three components:

\begin{enumerate}
	\item ViewMatchers - allows to find view in the current view hierarchy
	
	\item ViewActions - allows to perform actions on the views
	
	\item ViewAssertions - allows to assert state of a view
\end{enumerate}

\begin{android}
onView(ViewMatcher)       
.perform(ViewAction)     
.check(ViewAssertion);
\end{android}

If Espresso does not find a view via the ViewMatcher, it includes the whole view hierarchy into the error message. That is useful for analyzing the problem.

See the example in \href{https://github.com/googlesamples/android-testing}{The testing examples}


\subsection{ActivityTestRule}
he following section describes how to create a new Espresso test in the JUnit 4 style and use ActivityTestRule to reduce the amount of boilerplate code you need to write. By using ActivityTestRule, the testing framework launches the activity under test before each test method annotated with @Test and before any method annotated with @Before. The framework handles shutting down the activity after the test finishes and all methods annotated with @After are run.



\subsection{Hamcresting}
 The Hamcrest library provides us with common matchers and possibility to create custom matchers. Hamcrest provides a library of matcher objects (also known as constraints or predicates) allowing 'match' rules to be defined declaratively, to be used in other frameworks. Typical scenarios include testing frameworks, mocking libraries and UI validation rules.
 
 The base idea is that matcher is initialized with the expected values, which are compared against the actual object we are matching when invoking it.
 
 Among of the common matchers you can create your custom matchers using abstract class TypeSaveMatcher.class , with three methods to override:
 
 \begin{enumerate}
 	\item public boolean matchesSafely() - matcher's logic,
 	\item protected void describeMismatchSafely Description 
 \end{enumerate}

To see this at work, refer to the \href{https://github.com/googlesamples/android-testing/tree/master/ui/espresso/CustomMatcherSample}{CustomMatcherSample}.

\section{Working with data adapters}
AdapterViews such as list, grids and spinners are different than the usual layouts (e.g. LinearLayout) because they don’t keep all their child elements in the view hierarchy. The main purpose of AdapterViews is to show large data sets on the screen efficiently, so they have to optimize memory use and performance by only maintaining a View object for data elements that currently fit inside the viewport.
All other elements exist only as the data set in the Adapter that is backing the AdapterView. With Espresso.onData(Matcher dataMatcher) you supply a Matcher that will try to match a row in the Adapter. If there is a successful match, Espresso will then bring that row onto the screen and into the view hierarchy so that you can perform actions and check assertions on its view as usual:

\begin{android}
onData(Matcher dataMatcher)
.perform(ViewAction action)
.check(ViewAssertion assert)
\end{android}


So how does the data matcher that you supply to onData() work? Basically, Espresso goes through the items in the Adapter backing your AdapterView one by one, and passes the result of Adapter.getItem(int position) to the Matcher.

See the \href{https://github.com/googlesamples/android-testing/tree/master/ui/espresso/DataAdapterSample}{Data Adapter sample for Espresso}

\subsubsection{The case of the RecyclerView}


At first it might seem like the same method should be applied to a RecyclerView, as it also uses adapters to hold data and reuses a small amount of View objects to display that data on the screen. Unfortunately, RecyclerView does not inherit from AdapterView (it’s a direct subclass of ViewGroup instead), so you can’t use onData with it.
Instead, you should use one of the RecyclerViewActions methods to scroll the RecyclerView to the desired item and perform a ViewAction on it (using a ViewHolder matcher or position):

\begin{android}
onView(withId(R.id.myRecyclerView))
.perform(
RecyclerViewActions.actionOnItemAtPosition(0, click())
);
\end{android}

See the \href{https://github.com/googlesamples/android-testing/tree/master/ui/espresso/RecyclerViewSample}{RecyclerViewSample }

\subsection{Espresso UI recorder}
Android Studio provides an Run Record Espresso Test menu entry which allows you to record the interaction with your application and create a Espresso test from it.


\newpage
\section{Exercises}
\begin{exercise}
	Write 3 UI tests for the DotPict app and the WhoIsIt app. 1 test should use a custom Matcher.
	
\end{exercise}

\chapterimage{images/books.png}
\printbibliography

\printindex
\end{document}
